\documentclass[11pt,a4paper]{article}
%\usepackage[T1]{fontenc}
\usepackage[utf8]{inputenc}
\usepackage{geometry}   % Setup for page and paper dimensions
\usepackage{wrapfig}
\usepackage{caption}
\captionsetup{
  font=small,
  labelfont=bf,
  tableposition=top
}

\usepackage[swedish]{babel}
\usepackage{hyperref}
\hypersetup{
    colorlinks,
    citecolor=black,
    filecolor=black,
    linkcolor=black,
    urlcolor=black,
    pdfborder = {0 0 0}
}
%\usepackage{verbatim} 
%\usepackage{fancyvrb}
\usepackage{subcaption}
\usepackage{verbatimbox}
\usepackage{multicol}
\usepackage{graphicx}   
\usepackage{lipsum}  
\usepackage{amsmath}
\usepackage{listings}

%\renewcommand{\figurename}{Figur}
%\renewcommand{\tablename}{Tabell}

%\addto\captionsenglish{%
%  \renewcommand\tablename{Tabell}
%}
%\addto\captionsenglish{%
%  \renewcommand\figurename{Figur}
%}
%\renewcommand\figurename{Figur}
   
\title{Laborationer Termodynamik Grund}
\author{Alf Heidemark, Pontus Ösbjer, Peter Larsson, Lasse Karagiannis}
%\date{} % Activate to display a given date or no date (if empty),
         % otherwise the current date is printed 
 
\begin{document}
\maketitle
%\lipsum[1]
\section{Sammanfattning}
Laborationen bestod av två delar. Den första delen handlade om att utifrån
mätdata inhämtade från tre adiabatiska kompressioner bestämma $\kappa$ genom
kurvanpassning och den andra delen bestod i att analysera mätdata från en 
Sterling-maskin som kördes såsom värmemotor och som kylmaskin.
Avseende att göra anpassningen så rekommenderades Excel men vi valde att automatisera
arbetet genom ett optimeringsskript som hittades på ett matlabforum\footnotemark{}.
\footnotetext{\url{https://se.mathworks.com/matlabcentral/answers/344980-curve-fitting-loglog-plot}}
Resultatet av anpassningen av de tre mätförsöken som \bf{inte}\rm ~ses i figuren nedan och vi förklarar matlabkoden
längre ner i labbrapporten.
Avseende Sterlingmaskinen så var uppgiften att generera ett antal plottar som redovisas
i denna rapport.



\begin{figure}[h]
\centering
\begin{subfigure}{0.3\textwidth}
    \includegraphics[width=\textwidth]{fig1.eps}
    \caption{Mätserie 1: $\kappa =1.3815$}
    \label{fig:first}
\end{subfigure}
%\hfill
\begin{subfigure}{0.3\textwidth}
    \includegraphics[width=\textwidth]{fig2.eps}
    \caption{Mätserie 2: $\kappa=1.3848$}
    \label{fig:second}
\end{subfigure}
%\hfill
\begin{subfigure}{0.3\textwidth}
    \includegraphics[width=\textwidth]{fig3.eps}
    \caption{Mätserie 3: $\kappa=1.3866$}
    \label{fig:third}
\end{subfigure}
        
\caption{Anpassning av $n$ i $p=\frac{C}{V^n}$}
\label{fig:figures}
\end{figure}


\begin{multicols}{2}
\subsection{Anpassning av mätdata}
\bigskip
Koden för anpassningen är följande och upprepas
för att åstadkomma kurvanpassningen med plottar för repsektive
volym-tryck par därför visas endast koden för den första grafen.
Resterande åstadkommes med upprepning. Inläsningen och konditioneringen
av data visas dock fullständigt.

\subsubsection{Inläsningen av data från Excel}
\begin{verbnobox}[\tiny\arabic{VerbboxLineNo}\small\hspace{1ex}:\footnotesize]
%Inläsning
[num] = xlsread('matdata') ;
V1=num(1:25,1);
p1=round(151975.*V1.^(-1.331),1);
V2=num(1:25,4);
p2=round(208884.*V1.^(-1.39),1);
V3=num(1:25,7);
p3=round(245855.*V1.^(-1.42),1)
\end{verbnobox}

Här sker inläsning av den 25 första mätvärdena för kolumnerna \tt{V1}\rm, \tt{V2} \rm och \tt{V3}\rm
~från kolumnerna 1, 4 och 7 ur excel-filen.
Inspekterandes excel-dokumentet så såg vi att Matlabs inläsning av p-kolumnerna resulterade
i annorlunda decimalvärden än värdena i Excel, t.ex. 115.9 resulterade i 115.899233
så därför gjordes en avrundning til en decimal så att Matlabs värden skulle överensstämma
med de värden som gavs i Excel-dokumentet.

\subsubsection{Konditionering av data}
\begin{verbnobox}[\tiny\arabic{VerbboxLineNo}\small\hspace{1ex}:\footnotesize]
%Konditionering till m^3 och Pa
V1=V1.*10^(-6);
V2=V2.*10^(-6);
V3=V3.*10^(-6);
p1=p1.*10^(3);
p2=p2.*10^(3);
p3=p3.*10^(3);
\end{verbnobox}

Volymerna gavs i $\rm{cm}^3$ så därför multiplicerade vi $10^{-6}$ för att erhålla
$\rm{m}^3$ och vidare så gavs trycket i kP så därför multiplicerade vi med $1000$
för att få enheten i Pa.

\subsubsection{Kurvanpassning}
\begin{verbnobox}[\tiny\arabic{VerbboxLineNo}\small\hspace{1ex}:\footnotesize]
% Kurvanpassning
fit_fcn = @(b,V) V.^b(1).* exp(b(2));

RNCF = @(b) norm(p1 - fit_fcn(b,V1));                                   
B1 = fminsearch(RNCF, [1;1]); 
\end{verbnobox}

Kurvanpassningsfunktionen hittades på nätet. Den är onödigt kompakt och därför svår att förstå. Rad nr. \tt{2}\rm är en \tt{fit\_function}\rm är en pekare
till en anonym funktion genom symbolen \tt @.\rm ~Funktionen har argumentslistan
\tt{(b,V)}\rm där \tt{b}\rm ~är en vektor med två rader och en kolumn. Därefter följer
funktionskroppen som verkar måste stå på samma rad men detta är oklart.
Funktionskroppen utgörs av mönsterfunktionen som programmet skall försöka matcha till
\verb|V.^b(1).* exp(b(2))|
\noindent Inkluderar man inte exponentialfunktionen så kommer matchningen
att bli ca $1.38$ för samtliga fall, se \ref{fig:figures}.
Ytterliggare en funktionspekare definieras \tt{RNCF}\rm ~som endast tar ett enda argument
och där funktionskroppen  bildar normen av differensen mellan verkliga mätdata \tt{p1}\rm
~och de funktionsvärden som matlab försöker anpassa så att skillnaden blir så liten som möjligt.
Argumentet till \tt{RNCF}\rm ~blir en inparameter till anropet av \tt{fit\_function}\rm
~och där andra inparametern är volymsvektorn. Detta är den s.k. kostnadsfunktionen
~som nästa funktionsanrop söker minimera.
\tt{B1 = fminsearch(RNCF, [1;1])}\rm. ~Två inparametrar, den ena är pekaren RNCF och den andra
är startvärden för optimeringen.
Utdata från  för Matlabsamtliga tre körningar ges nedan där Matlabs radbrytningar har editerats bort.
\begin{verbnobox}[\footnotesize]
B1 =
   -1.3310
    0.4510
B2 =
   -1.3901
   -0.0469
B3 =
   -1.4202
   -0.2998
\end{verbnobox}
Första raden i varje kolumn är det eftersökta $\kappa$. Optimeraren har funnit
att den funktion som best mappar förhållandet mellan $p$ och $V$ är funktionerna
\begin{flalign}
       	p_{1}&=\frac{e^{0.45}}{V^{1.331}_{1}} & \\
        p_{2}&=\frac{e^{-0.0469}}{V^{1.3901}_{2}}&\\
        p_{3}&=\frac{e^{-0.2998}}{V^{1.4202}_{3}} 
\end{flalign}

\subsubsection{Plottgenerering}
\begin{verbnobox}[\tiny\arabic{VerbboxLineNo}\small\hspace{1ex}:\footnotesize]
%plott
figure(1)
plot(V1, p1, 'pg')                                                       
hold on
plot(V1, fit_fcn(B1,V1))                                          
hold off
grid 
B1
\end{verbnobox}
Optimerade heldragna kurvan och den diskreta kurvan.
\subsubsection {Resultat av anpassningen}
Vi tycker att vi har fått bra överensstämmelse med exponenterna som användes
att generera datan.
Medelvärdet med korrekt antal värdesiffror är 1.38.
 \end{multicols}
\section{Stirlingmaskinen}
Sterlingmaskinen kördes dels som värmemotorprocess och dels som kylmaskin.
\subsection{Värmemotorprocess}

\begin{table}[h!]
\centering
\begin{tabular}{||c c c c c||} 
 \hline
 M/mNm & n/rpm & T1$/^{o}$C &T1$/^{o}$C& $\rm{W_{pV}}$/mJ\\ [0.5ex] 
\hline\hline
0	& 982 &	163	& 74.8 & 200 \\
2 &	945 &	169 &	77.7 &	201 \\
4	& 908 &	168 &	78.7 &	205 \\
6	& 860 &	177 &	77.5 &	210 \\
8	& 817	& 177	& 77.1 &	216 \\
10 &	745 &	178 &	76.5 & 221 \\
12 &	752&	179&	76.3&	230\\
14&	705&	185	&76.7&	238\\
15&	650&	188&	76.9&	239\\
17&	519&	190&	76.3&	243\\
18&	555&	192&	75.5&	245\\
20&	460&	195&	74.2&	246\\
22&	380&	197&	72&	247\\
23&	275&	201&	70.7&	235\\ [1ex] 
 \hline
\end{tabular}
\caption{Givna mätdata värmemotorprocess}
\label{table:1}
\end{table}
Utifrån givna mätdata enligt Tabell\ref{table:1} så skulle följande parametrar
räknas ut. 
$\rm{W_m}= 2\pi\cdot\rm{M}$ där M är det uppmätta momementet.
Motormoment balanserar mot friktionsmomentet och har alltid enheten energi.
$\rm{W_m}$ är det arbetet som motornaxeln åstadkommer under ett varv.
Multiplikation av M med $2\pi$ radianer ändrar inte enheten hos $\rm{W_m}$ 
därför att radianer har definierats såsom enhetslöst.
Vi gör detta för att momentet förhåller sig till effekten P enligt $P=T\cdot\omega$
vilket är den fjärde kolumnen.

 $\rm{W_{pV}}$ är volymändringsarbetet , det arbetet som omslutes av kretskurvan.
Det tekniska arbetet $W_{t}$ förhåller sig till friktionsarbetet $\rm{W_{fr}}$ enligt
\begin{equation}
W_{t} = \rm{W_m} + \rm{W_{fr}}
\end{equation}

Slutligen beräknades verkningsgraden såsom Carnotprocess $\eta_c$ 
enligt standardformeln.
\begin{equation}
\eta_c = 1-\frac{T_2}{T_1}
\end{equation}

\begin{table}[h!]
\centering
\begin{tabular}{||c c c c c ||} 
 \hline
$\rm{W_m}$/mJ & $\rm{W_{fr}}$/mJ & f/Hz & $\rm{P_m}$/mW & $\eta_c$\\ [0.5ex] 
\hline\hline
	0	&200&	16.37&	0&	0.20\\
12.5&	194.72&	15.75&	98.96&	0.207\\
25.1&	179.87&	15.13&	380.34&	0.202\\
37.7&	191.15&	14.33&	270.18&	0.221\\
50.3&	165.73&	13.62&	684.45&	0.222\\
62.8&	189.58&	12.42&	390.08&	0.225\\
75.40&	154.60&	12.53&	944.99&	0.227\\
87.9&	194.02&	11.75	&516.79&	0.236\\
94.3&	144.75&	10.83&	1021.02&	0.241\\
106.81&	189.59&	8.65&	461.97&	0.246\\
113.10&	131.90&	9.25&	1046.15&	0.251\\
125.7&	183.17&	7.67&	481.71&	0.258\\
138.2&	108.77&	6.33&	875.46&	0.266\\
144.5&	162.74&	4.58&	331.18&	0.275\\[1ex] 
 \hline
\end{tabular}
\caption{Beräknade parametrar från mätdata värmemotorprocess}
\label{table:2}
\end{table}
%    \lipsum[2]
%    \bigskip
%    \noindent   % Very important in case that \parindent is not 0pt
%    \begin{minipage}{\linewidth}
%      \centering
%      \includegraphics[width=3.5cm]{fig1.eps}
%      \captionof{figure}{Figure caption}
%      \label{fig:cc}
%    \end{minipage}
%    \lipsum[3]

Uppgiften var att generera två plottar där den ena innehåller de tre
graferna för 
\begin{figure}[h]
\centering
\begin{subfigure}{0.45\textwidth}
    \includegraphics[width=\textwidth]{Stirlingmotor.eps}
    \caption{ $\rm{W_m}$/mJ, $\rm{W_{fr}}$/mJ och $\rm{W_{pV}}$/mJ}
    \label{fig:first}
\end{subfigure}
%\hfill
\begin{subfigure}{0.45\textwidth}
    \includegraphics[width=\textwidth]{eta.eps}
    \caption{Karnotverkningsgraden $\eta_c$}
    \label{fig:second}
\end{subfigure}

\caption{Stirlingsmotorns karaktäristik}
\label{fig:fig2}
\end{figure}

\begin{multicols}{2}
\vfill\null
\columnbreak

\subsection{Stirlingmaskinen såsom kylprocess}
Avseende Stirlingmaskinen såsom kylmaskin så blev resultatet
att varvtalet 240 rpm gav temperaturerna $T_{1}=34^{\rm{o}}\rm{C}$
och $T_{2}=17^{\rm{o}}\rm{C}$. Köldfaktorn $\epsilon$ är formelmässigt inversen av
$\eta$ och om vi får anta att Stirlingmaskinen är lika effektiv som kylmaskin som den är 
som värmemotorprocess så skulle vi kunna approximera $\epsilon$ trots att vi inte har
några mätdata för 240 rpm. Om vi tar $\eta_{c}(240\rm{rpm})\approx 0.25$
så får vi $\epsilon = 4$.

Om vi använder formeln för köldfaktorn $\epsilon$ så skulle vi kunna
säga något den bortförda värmen $Q_{bort}$
\begin{equation}
\epsilon = \bigg|{\frac{Q_{bort}}{W_{t}}}\bigg|
\end{equation}


Vi vet att $W_{t}=W_{pV}$ så om vi tar värdet som vi har för 275 rpm
och extrapolerar det till 240 rpm som höftas till 220 mJ så
får ${Q_{bort}}=4\cdot 0.22 =0.88 J$.Med detta värde kanske vi kan säga något
om volymen på isolerboxen.Vi gissar rumstemperaturen till $22^{\rm{o}}\rm{C}$
Densiteten $\rho$ för luft är 1.2kg/$\rm{m}^3$ och isolerboxens process sker
under konstant tryck.$c_p$ för luft är 1010 J/$\rm{kg}\cdot\rm{K}$.
Isolerboxens process måste kunna ses som ett öppet system eftersom
varm luft diffunderar in och blandas med den kalla luften där vissa luftmolekyler
läcker ut genom springor. Inget arbete sker, $c_1$, $c_2$ och lägesenergi
försummas i Bernoulliformuleringen av TD1. 

\begin{flalign*}
       Q_{bort} &=I_2-I_1 &\\
       Q_{bort} &= m\cdot c_{p}\cdot(T_{rum}-T_{kall}) &\\
       0.88&= \rho\cdot V\cdot (22-17) &\\
      0.88&=1.2\cdot V \cdot 1010\cdot 5
\end{flalign*}
Vilket ger $V=0.88/(1.2\cdot 5\cdot 1010 )=0.000145215 \rm{{m}^3}$.
Tredje roten ur ger ett sidomått på 0.052 m dvs ca 5 cm.
Var är felet lärare! Är det ett teoretiskt misstag eller är det
inverkande faktorer som inte medtagits?

\end{multicols}

\listoffigures
\listoftables
\tableofcontents

\end{document}